\subsubsection{C7 Pregunta 9}
\begin{lstlisting}
import java.lang.Math;

public class Cap7ej9 {
	
	public static void main (String args[]) {
		long tiempoinicial, tiempoflotante=0, tiempoentero=0, tiempofinalflotante=0, tiempofinalentero=0;
		double valor = 0;
		long valorInt= 0;
		double valorFloat= 0;
		valor = Math.random()*1000000000;
		valor = valor*1000000;
		valorInt = (long)valor;
		valorFloat = valor;
		tiempoinicial = System.nanoTime();
		valorInt = operacionEntero(valorInt);
		tiempoentero = System.nanoTime();
		valorFloat = operacionFlotante(valorFloat);
		tiempoflotante = System.nanoTime();
		tiempofinalentero =  tiempoentero - tiempoinicial;
		tiempofinalflotante = tiempoflotante - tiempoinicial;
		System.out.println("tiempo long int: "+tiempofinalentero+"\ntiempo double float: "+tiempofinalflotante);
			}
	private static double operacionFlotante(double valor){
		return (double)(valor+valor-valor*valor*valor*Math.pow(valor,100)/valor*Math.pow(valor,100));
		}
	private static long operacionEntero(long valor){
		return (long)(valor+valor-valor*valor*valor*Math.pow(valor,100)/valor*Math.pow(valor,100));
		}
}
 \end{lstlisting}

El programa realiza operaciones para entero y para flotante, siendo estos numeros demasiado grandes, lo que se pide es calcular el tiempo de respuesta al aplicar operaciones aritmeticas a flotante y entero. Los tiempos de respuesta para enteros fue menor a los tiempos de respuesta de flotantes, porque la precision es menor en enteros, mientras flotantes ocupa mas capacidad de memoria para acumular sus valores.
